\documentclass{article}
\usepackage{graphicx}
\usepackage{amssymb,amsmath,latexsym}
\usepackage{listings,microtype}
\usepackage{geometry}
\geometry{legalpaper, margin=1in}
\lstset{
  basicstyle=\ttfamily,
  columns=fullflexible,
  frame=single,
  breaklines=true,
  %postbreak=\mbox{\textcolor{red}{$\hookrightarrow$}\space},
}
\usepackage{comment}
\title{Developing accurate KKT formulation for NLP systems}

\begin{document}

\section{Introduction}
Solving constrained non linear optimization problems begins with identification of the Karush-Kuhn-Tucker conditions due to the fact that for a differentiable convex system where constraint qualification holds, the solution of NLP is the solution to the KKT system and vise versa. The KKT conditions enables solving the NLP model as an MCP model by representing the KKT conditions as complimentarily equations.  Thus, it is often advantageous to explicitly model the KKT conditions as solve them as a Mixed Complementarity Problem (MCP), as described in the PATH Solver Manual (/link). The example provided is of a Linear Transportation model where the traditional model of fixed demand and price is made more realistic by making the variables endogenous, i.e. the price of commodity is market affects the demand of the commodity.  However, at times the complexity of the model causes errors in formulation of KKT conditions, leading to incorrect or infeasible solutions. Finding the source of the error can be a particularly tedious task. Thus, in this article, we propose a systematic method for identification of errors in the explicit KKT conditions for solving an NLP. . 

We begin by understanding the definition of complementarity. If a function $F(z) : {\!R}^n \mapsto {\!R}^n$, lower bounds $ l \in { \!R \cup {-\infty}}^n$ and upper bounds $ u \in { \!R \cup {\infty}}^n$ has a solution vector $z \in \!R$ such that for each $ i \in {1,...,n}$, one of the following three conditions hold : 

 \centerline{$F_{i}(z) = 0$  and  $ l_i \leq z_i \leq u_i $   or} 
 \centerline{$F_{i}(z) > 0$  and  $ z_i = l_i$  or }
 \centerline{ $F_{i}(z) < 0$  and  $ z_i = u_i$ }

then function $F$ is complementary to the variable $z$ and its bounds. This is written in compact form as 

\centerline{ $ F(z) \perp L \leq z \leq U $ }
\par
\noindent Where the symbol $\perp$ means "perpendicular to".  From point of view of optimization, if $F(z)$ is non zero (non-binding constraint), changes in $z$ may further optimize the objective function until the constraint becomes binding. If $F(z)$ is binding, no changes in $z$ would further enhance the objective function, causing the marginal to be zero. 

Thus for a general NLP problem,

\begin{equation}
\begin{aligned}
&	\min
& & f(x) \\
& \text{s.t.} & & 	 g_{i}(x) \leqslant 0	&	i = 1,2...m \\
& & &			h_{k}(x) = 0	 &	k = 1,2...p \\
& & &			d_{l}(x) \geqslant =0		&	l = 1,2...q \\
& & &			L \leq x \leq Z \\
& & &			f(x): {\!R}^n \mapsto \!R , g(x): {\!R}^n \mapsto {\!R}^m\\
& & &			h(x): {\!R}^n \mapsto {\!R}^ p , d(x): {\!R}^n \mapsto {\!R}^ p\\
\end{aligned}
\end{equation}

\noindent the Lagrange function and KKT conditions in the complementarity form are written as \\ 

\begin{equation}
\begin{aligned}
& L(x,u,v,w) = f(x) - <u,g(x)> - <v,h(x)> - <w,d(x)>  \\
& \bigtriangledown_x L  \perp L_x \leq x \leq U_x 	\\
& - \bigtriangledown_u L  \perp u \geq 0	\\
& -\bigtriangledown_v L  \perp v free	\\	
& -\bigtriangledown_w L  \perp w \leq 0	\\
\end{aligned}
\end{equation}

The gradient w.r.t to marginals result in the inequality and equality constraints of the NLP.





%--------------------------------------%--------------------------------------%--------------------------------------%--------------------------------------
\begin{comment}

\begin{equation}
\begin{aligned}
& \bigtriangledown{f(\hat{x})} - \sum_{i=1}^{m} u_{i} \bigtriangledown{g_{i}(\hat{x})}
			- \sum_{k=1}^{p} v_{i} \bigtriangledown{h_{k}(\hat{x})} - \sum_{l=1}^{q} w_{i} \bigtriangledown{d_{l}(\hat{x})} = 0  \\
\\
& h_{k}(\hat{x}) = 0   k = 1,2...p  \\
g_{i}(\hat{x}) \leq 0&	 i = 1,2...m \\  d_{l}(\hat{x}) \geq =0	&	l = 1,2...q
\\
and,\\
<u_{i},g_{i}(x)> = 0 \\ <v_{i},h_{k}(x)> =0 \\  <w_{l},d_{l}(x)> =0
\end{aligned}
\end{equation}

where $<u_{i},g_{i}(x)> = 0$  represent the complimentarily condition and variables u, v, and w represent the marginals of the respective constraint. It is often written as 

 $g_{i}(x) \perp L \leq u \leq U $

where symbol $\perp $(referred to as perpendicular  to) indicates pair-wise complementarity between the function g() and variable u and its bounds. The complimentairy condition essentially 

\end{comment}
%--------------------------------------%--------------------------------------%--------------------------------------%--------------------------------------

Thus an NLP can be solved as an MCP using the KKT conditions. However, formulating the KKT conditions might prove difficult for complex systems, and need verification of the accuracy. In the following sections, we provide the framework for modeling the KKT conditions  by using concept of dummy complementarity equations in the KKT formulation. The process includes the following steps:

\begin{enumerate}
	\item Solve NLP without explicit KKT conditions, and save the results
	\item	 Restart the problem as an MCP with dummy KKT conditions using solution from step 1 as initial point. The MCP should start at the solution itself
	\item Replace one dummy equation with one KKT condition. Resolve the NLP and save the results
	\item Restart the problem as MCP. If MCP iteration count is > 0, there exists a problem with the KKT condition which was introduced.
\end{enumerate}
\noindent Follow steps 3 and 4 until all KKT conditions have been successfully incorporated.
  

\section{Example : Maximum Revenue- NLP formulation}

Consider a simple case of a steel factory trying to maximize the revenue under budget constraints, with man-hours(h) and raw materials steel (s) as the decision variables. The revenue is a function of decision variables given by 

\centerline{$R(h,s) = 200 h^{(2/3)}s^{(1/3)} $ }
\bigbreak
\noindent where, budget = \$ 20,000
cost of manpower = \$ 20 /hr
cost of raw material = \$ 170 / tonn 

\noindent In it's standard form, the model can be written as :

\begin{equation}
\begin{aligned}
&	\min
& & R(h,s) = - 200 h^{2/3}s^{1/3}  \\
& \text{s.t.} & & 	 20h + 170s \leq 20000 \\
& & &			L< h,s < U   \\
\end{aligned}
\end{equation}

The GAMS program below gives the optimum value of revenue R at $h = 666.67 , s = 38.21$.

\lstinputlisting {codes/sd_nlp.gms}

\noindent The above model can we written in form of an MCP, by explicitly adding the KKT conditions to the model. For the given system, the KKT conditions are given as

\begin{equation}
\begin{aligned}
 L = & - 200 h^{2/3}s^{1/3} - con1_m [ 20h + 170 s - 20000]	\\
 \bigtriangledown _h L = & - 200* (2/3) h^{(-1/3)}  s^{(1/3)} - con1\_m*(20)  	\\ 
 \bigtriangledown _s L:  & - 200 * (1 / 3) h^{(2/3)} *(1/3) *  s^{(-2/3)} - con1\_m*(170)   \\
 \bigtriangledown _{con1_m} L : &   20*h + 170 * s - 20000 =0 \\
\end{aligned}
\end{equation}

\noindent The third KKT equation is the inequality budget constraint itself. The above equations are solved as an MCP model as shown below.

\lstinputlisting{codes/sd_kkteq.gms}


\end{document}

