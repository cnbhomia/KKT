\documentclass{article}
\usepackage{graphicx}
\usepackage{amssymb,amsmath,latexsym}
\usepackage{listings}
\lstset{
  basicstyle=\ttfamily,
  columns=fullflexible,
  frame=single,
  breaklines=true,
  %postbreak=\mbox{\textcolor{red}{$\hookrightarrow$}\space},
}
\usepackage{comment}
\title{Developing accurate KKT formulation for NLP systems}

\begin{document}

\section{Introduction}
The modern solving strategies for solving constrained non linear optimization problems begin with identification of the Karush-Kuhn-Tucker conditions. Several NLP solvers take advantage of the fact that for a differentiable convex system where constraint qualification holds, the solution of NLP is the solution to the KKT system and vise versa.  For example PATHNLP (/link) derives the KKT system of an NLP model and solves it as a MCP system.  However, it is sometimes advantageous to explicitly model the KKT conditions as part of the NLP problem formulation. At time, the complexity of the model leads to error in deriving the KKT conditions, leading to incorrect or infeasible solutions. Finding the source of the error can be a particularly tedious task.
In this article, we propose a systematic method for identification of errors in the explicit KKT conditions for solving an NLP. Consider a constrained optimization problem such that 

\begin{equation}
\begin{aligned}
&	\min
& & f(x) \\
& \text{s.t.} & & 	 g_{i}(x) \leqslant 0	&	i = 1,2...m \\
& & &			h_{k}(x) = 0	 &	k = 1,2...p \\
& & &			d_{l}(x) \geqslant =0		&	l = 1,2...q \\
& & &			x \in \!R
\end{aligned}
\end{equation}

The Karush Kuhn Tucker (KKT) conditions provide the necessary conditions for a local minimum at $\hat{x}$ :

\begin{equation}
\begin{aligned}
& \bigtriangledown{f(\hat{x})} - \sum_{i=1}^{m} u_{i} \bigtriangledown{g_{i}(\hat{x})}
			- \sum_{k=1}^{p} v_{i} \bigtriangledown{h_{k}(\hat{x})} - \sum_{l=1}^{q} w_{i} \bigtriangledown{d_{l}(\hat{x})} = 0  \\
\\
& h_{k}(\hat{x}) = 0   k = 1,2...p  \\
g_{i}(\hat{x}) \leq 0&	 i = 1,2...m \\  d_{l}(\hat{x}) \geq =0	&	l = 1,2...q
\\
and,\\
<u_{i},g_{i}(x)> = 0 \\ <v_{i},h_{k}(x)> =0 \\  <w_{l},d_{l}(x)> =0
\end{aligned}
\end{equation}

where $<u_{i},g_{i}(x)> = 0$  represent the complimentarily condition and variables u, v, and w represent the marginals of the respective constraint. It is often written as 

 $g_{i}(x) \perp L \leq u \leq U $

where symbol $\perp $(referred to as perpendicular  to) indicates pair-wise complementarity between the function g() and variable u and its bounds. The complimentairy condition essentially 
The process of solving NLP with explicit KKT conditions has two major step

\begin{enumerate}
	\item Checking for errors in KKT by solving NLP as an MCP
	\item Using dummy equations for KKT system to identify equations with errors
\end{enumerate}
Using dummy equations to identify the error(s) with KKT system is discussed in subsequent section. 

\begin{comment}
%{
\begin{enumerate}
	\item Solve NLP without explicit KKT conditions
	\item	 Solve problem as MCP using the KKT conditions with solution from step 1 as initial point
	\item If MCP iteration count is > 1, there exists a problem with one of the KKT conditions.
\end{enumerate}
Identification and tracing of the error is covered in subsequent sections
%}    
\end{comment}

\section{Maximum Revenue- NLP formulation}

Consider a simple case of a steel factory trying to maximize the revenue under budget constraints, with man-hours(h) and raw materials steel (s) as the decision variables. The revenue is a function of decision variables given by \\

\centerline{$R(h,s) = 200 h^{(2/3)}s^{(1/3)} $ }
\bigbreak
\noindent where, budget = \$ 20,000
cost of manpower = \$ 20 /hr
cost of raw material = \$ 170 / tonn \\

\noindent In it's standard form, the model can be written as :

\begin{equation}
\begin{aligned}
&	\min
& & R(h,s) = - 200 h^{2/3}s^{1/3}  \\
& \text{s.t.} & & 	 20h + 170s \leq 20000 \\
& & &			L< h,s < U   \\
\end{aligned}
\end{equation}

The GAMS program below gives the optimum value of revenue R at $h = 666.67 , s = 38.21$.

\lstinputlisting {codes/sd_nlp.gms}

\noindent The above model can we written in form of an MCP by explicitly adding the KKT conditions to the model. For the given system, the KKT conditions are given as

\begin{equation}
\begin{aligned}
& dLdh : & - 200* (2/3) h^{(-1/3)}  s^{(1/3)} - con1\_m*(200) = 0 	 \\
& dLds:  & - 200 * (1 / 3) h^{(2/3)} *(1/3) *  s^{(-2/3)} - con1\_m*(170)    = 0 	\\
& con1 :   & 20*h + 170 * s = 20000 \\ 
&  & con1\_m *con1 = 0 \\
\end{aligned}
\end{equation}

\noindent where \textit{con1\_m} is the marginal, or Lagrange multiplier of the constraint \textit{con1} , and \textit{dLdh},\texit{dLds} are the gradients of the Lagrange function on the model. 

The original model with KKT  can be reformulated as an MCP in accordance to the complementarity conditions described in Section 1. 

\begin{equation}
\begin{aligned}
&  \frac{dL}{dh} \perp h = 0 	 \\
& \frac{dL}{ds} \perp s = 0 	\\
& con1\_m \perp con1 = 0 \\
\end{aligned}
\end{equation}


\end{document}
