\documentclass{article}
\usepackage{graphicx}
\usepackage{amssymb,amsmath,latexsym}
%\userpackage{textcomp}
\title{Developing accurate KKT formulation for NLP systems}

\begin{document}

\section{Introduction}
The modern solving strategies for solving constrained non linear optimization problems begin with identification of the Karush-Kuhn-Tucker conditions. For a given NLP model, the solvers create the KKT conditions internally and provide the Lagrangian Multipliers as marginals as part of the solution. However, it is sometimes advantageous to explicitly model the KKT conditions as part of the problem formulation. However, the complexity of the models create possibilities of error in manually deriving the KKT conditions , leading to incorrect or infeasible solutions. Finding the source of the error can be a particularly tedious task. 

In this article, we propose a systematic method for identification of errors in the explicit KKT conditions for solving an NLP. Consider a constrained optimization problem such that 

\begin{equation}
\begin{aligned}
&	\min 
& & f(x) \\
& \text{s.t.} & & 	 g_{i}(x) \leqslant 0	&	i = 1,2...m \\
& & &			h_{k}(x) = 0	 &	k = 1,2...p \\
& & &			d_{l}(x) \geqslant =0		&	l = 1,2...q \\
& & &			x \in \!R
\end{aligned}
\end{equation} 



The Karush Kuhn Tucker (KKT) conditions provide the necessary conditions for a local minimum at $\hat{x}$ :

\begin{equation}
\begin{aligned}
& \bigtriangledown{f(x)} - \sum_{i=1}^{m} u_{i} \bigtriangledown{g_{i}(x)} 
			- \sum_{k=1}^{p} v_{i} \bigtriangledown{h_{k}(x)} - \sum_{l=1}^{q} w_{i} \bigtriangledown{d_{l}(x)} = 0  \\
\\
& h_{k}(x) = 0   k = 1,2...p  \\	
g_{i}(x) \leqslant 0&	 i = 1,2...m \\  d_{l}(x) \geqslant =0	&	l = 1,2...q
\\
and,\\
<u_{i},g_{i}(x)> = 0 \\ <v_{i},h_{k}(x)> =0 \\  <w_{l},d_{l}(x)> =0
\end{aligned}
\end{equation}

where $<u_{i},g_{i}(x)> = 0$  represent the complimentarily condition, with u, v, and w representing the marginals of the respective constraint.

The process of solving NLP with KKT conditions has two major step

1. Identifying if an error exists
2. If yes, trace the error

Identifying if an error exists has the following steps :
1. Solve NLP without explicit KKT conditions
2. Solve problem as MCP using the KKT conditions with solution from step 1 as initial point
3. If MCP iteration count is > 1, there exists a problem with one of the KKT conditions.

Identification and tracing of the error is covered in subsequent sections



\section{Maximum Revenue- NLP formulation}

Consider the problem where one is running a steel factory under budget constraints(B) with man-hours(h) and raw materials steel (s) as constraints. The revenue is a function of decision variables as 

$R(h,s) = 200 h^(2/3)s^(1/3) $
where, budget = \$ 20,000 
cost of manpower = \$ 20 /hr
cost of raw material = \$ 170 / tonn

In it's standard form, the model can be written as :

\begin{equation}
\begin{aligned}
&	\min 
& & R(h,s) = - 200 h^{2/3}s^{1/3}  \\
& \text{s.t.} & & 	 20h + 170s = 20000 \\
& & &			h > 0	 &	s > 0 \\
\end{aligned}
\end{equation} 

The GAMS program below gives the optimum value of revenue at $h = 666.67 , s = 38.21$.






\end{document}